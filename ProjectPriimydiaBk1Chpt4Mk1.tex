\documentclass[10pt]{article}
\usepackage[margin=1in]{geometry}
\usepackage{setspace}
\usepackage{parskip}
\usepackage{lmodern}
\usepackage{titlesec}
\titleformat{\section}{\normalfont\Large\bfseries}{\thesection.}{1em}{}

\begin{document}

\begin{center}
    \Large\textbf{Book I: The One and the Five} \\
    \large Chapter 4: The Oath and the Divide \\
\end{center}

\vspace{1in}


\begin{center}
\begin{enumerate}
    \item \textbf{The Petition for a Crown} \\
    As divine favor diverges and miracles compete, factions form. One group calls on Priotheer to return as king — not of memory, but of fire. He refuses. Another voice rises.

    \vspace{1in}
    \item \textbf{The Fracturing of The Tree-Paths} \\
       The concentric structure of the city splits — physically and ritually. Altars override spiral order. Streets no longer echo. A section of the city secedes under a god-favored faction. A temple is defaced.
    \vspace{1in}
    \item \textbf{The First god-war of Words} \\
      A council is called — not by Priotheer, but by a new leader. Debate turns to accusation. Priests of Memory are publicly rebuked. Chant becomes chant becomes weapon. The gods begin to contradict each other.
    \vspace{1in}
    \item \textbf{The Fire-Walker's Oath} \\
    Priotheer, in private grief, walks the Tree one final time. He meets an orphaned follower who still remembers silence. Together they forge an oath — not of rule, but of protection. This becomes the first vow of the Fire-Walkers.

    \vspace{1in}
    \item \textbf{The Division of the Stone} \\
    One of the Tree’s outer branches breaks. A temple collapses. Priimydia is no longer whole. Those who follow the gods begin their own city. Those who follow the Pattern are left in ruin — or quiet resolve.

\end{enumerate}
\end{center}

\newpage

\section*{Segment 1: The Petition for a Crown}

They came not with torches, but with garlands.

Not with chants, but with requests.

At dawn, a procession formed outside the Temple of Earth —  
not circular, but linear.  
Each step broke from the city's original spiral.  
Each footfall aligned with desire, not design.

At its head stood Anytus, draped in blue ashcloth and silver flame.  
A former stone-caster. Now a voice people gathered around.  
He had not declared himself a prophet.  
But others had.

They reached Priotheer's dwelling beneath the western hollow.  
He was already waiting.

Anytus bowed.

 “The people are afraid.  
 The gods speak, but they speak to many.  
 And we need one who remembers the Pattern.”

Priotheer did not answer.

 “We do not ask for your silence.  
 We ask for your flame.”

Behind him, children held braziers lit with Isgrine’s fire.  
A statue of Paludin had been rolled on wheels.  
One boy carried a slab carved with Orson’s verse:  
 “What is owed must be returned.”

Anytus stepped forward.

 “We ask you to wear the crown again.  
 Not to rule as before.  
 But to guide in name of the gods.  
 The Meletheia is cracked.  
 The winds change mid-breath.  
 And the people hunger for direction.”

Priotheer looked at the stone beneath his feet.  
He touched it.  
It did not sing.

 “A crown forged by hunger will not lead to harmony,” he said.  
 “The gods have given.  
 But I was shaped to listen.”

Anytus’s expression did not change.  
He stepped aside.

 “Then you leave us to someone else.”

He did not threaten.  
He simply turned.

The procession followed.  
Straight paths. Forward steps.

No spiral.  
No refrain.

And Priotheer stood alone again.  
But this time, not as guardian.  
As refusal.


\newpage

\section*{Segment 2: The Fracturing of The Tree-Paths}

It began with one wall.

A merchant built it behind his altar to Isgrine,  
squaring off his corner of the marketplace.  
He said it was for safety.  
But it broke the spiral.

Then others followed.

Gardens were fenced.  
Spines of stone cut across the Tree-paths.  
Streets that once curved inward to the Stone Tree  
now jutted toward altars like roots split and searching.

The Priests of Memory marked the changes in silence.  
They walked slower.  
Some stopped walking altogether.

The Temple of Wind was relocated — not by priestly decree,  
but because a shrine to Aerun was better placed for trade winds.

The geometry no longer sang.  
It argued.

At dawn one morning, the western quarter raised a stone arch  
bearing all five god-symbols woven together.  
Beneath it, a plaque read:

 “We are One through the Many.”

No chant accompanied the arch.  
No alignment had been calculated.  
But people cheered.

By dusk, Priimydia’s spiral could no longer be walked without interruption.  
Stones that once echoed names now returned nothing.  
A new plaza formed — flat, angular, civic.

Priotheer stood at its edge.  
Children played in chalk markings shaped like flames and wings.

He did not interrupt them.

A former architect approached.

 “This is what you meant to prevent, isn’t it?”

Priotheer said:

 “No. This is what we were always going to choose.”

That evening, a torch procession crossed the Tree’s northern ring.  
They carried banners bearing the mark of Anytus:  
a line split by a single flame.

They did not announce a secession.  
They simply stopped returning to the inner circles.

And the stars, that night, seemed out of place.

\newpage

\section*{Segment 3: The First god-war of Words}

The Circle of Listening had once been a place of resonance.  
Now it echoed with interruption.

They gathered — not by call, but by necessity.  
Priests of Memory.  
Interpreters of the gods.  
Artisans. Children of flame.  
Even those who still sang alone.

No chant opened the council.  
Only Anytus, rising with a scroll in one hand and a flame in the other.

 “We must stop pretending memory is enough.  
 The gods have spoken.  
 The Pattern has moved.  
 If we do not move with it, we will be forgotten — or broken.”

A priest stood, robed in earth-brown and wind-stripes.

 “The gods cannot contradict the Pattern.”

Anytus replied:

 “They do not contradict it. They are it, awakened.”

The hall stirred.

Another voice — from the east side, where builders sat:

 “Then why does Aerun’s breath call for bridges,  
 while Orson’s law calls for walls?”

A silence followed.  
Then more questions.  
Not shouted — but offered like knives:

 “Why does Paludin’s grief silence song,  
 when Isgrine’s flame demands chorus?”

 “Why does Inanius whisper to children,  
 but fall quiet before the Meletheia?”

The Circle split.  
Not in violence.  
In posture.  
Some rose and walked clockwise.  
Others counter.

Anytus did not sit.

 “There is no longer one Pattern.  
 There are Five.  
 And through them — choice.”

He turned to Priotheer, who stood at the edge.

 “Say something. Before silence becomes surrender.”

Priotheer looked not at him,  
but at the stone under their feet.

 “The Pattern is not a law.  
 It is a song.  
 And songs do not divide.  
 They are divided.”

 “That is not the same,” Anytus said.

 “No,” said Priotheer. “It is not.”

No vote was taken.  
No war was declared.

But three Meletheia were struck from the wall that night.  
And in the morning, their words had not returned.

\newpage

\section*{Segment 4: The Fire-Walker's Oath}

Priotheer did not return to the Circle.

He walked alone, past the Tree’s forgotten roots,  
to the outer path — the first road, laid not by decree, but by faithful wandering.

There, in a hollow where silence had once taught him, he knelt.

He did not speak at first.

He placed his hands upon the dust.  
He lowered his brow to the ground.  
And only then, as dusk burned the stone in amber light, did he whisper:

 “I will not build altars.  
 I will not call gods by name.  
 I will not lead where longing shouts.  
 I will walk where fire still listens.”

Behind him — a faint rhythm.  
Not footsteps. A breath. A stillness.

A girl, no more than nine, stood watching.  
She had followed him, but said nothing.

In her hand, a char-marked reed — burned at one end,  
once part of the sacred spirals laid by the Priests of Memory.  
She pressed it gently into the earth beside him.

 “It used to hum when I touched it,” she said.  
 “Now it’s quiet.”

He looked at the mark on the reed.  
Then he gathered dry brush,  
and struck flint beside it —  
once.  
Twice.  
Until a small fire caught.

The reed warmed.  
It did not hum.  
But the silence it left was clean.

He turned to her.

 “Would you keep this fire?”

She nodded.

He placed the reed back in her hands.

Then he said:

 “Remember not what must be done.  
 Remember what must not be forgotten.”

She repeated it — once.

And that was the oath.

No title was given.  
No sign etched.  
No stone lifted.

But that night, a figure walked the spiral paths again,  
with fire in hand  
and silence at their side.

And another followed.  
And another.

And though no Meletheia rang,  
the dust beneath their feet remembered.

\newpage

\section*{Segment 5: The Division of the Stone}


It did not crack like thunder.

It sounded like breath held too long —  
then let go.

The Stone Tree, at its lowest arc,  
where roots met spiral,  
split.

Not shattered. Not destroyed.

But a long, quiet break —  
hairline at first,  
then widening with each season of silence.

No one claimed it.  
No voice took credit.  
No faction declared victory.

But the ground shifted.

The outermost temple — once the Chamber of Listening — tilted by two degrees.  
The wells beneath Paludin’s shrine emptied in half a day.  
Wind through Aerun’s arch blew sideways.

And then, quietly, without herald or decree,  
a new city began to build itself on the other side of the split.

Stone by stone,  
in corners,  
with altars already in place.

It was not called secession.  
It was called “Elsewhere.”

Priotheer stood at the faultline.  
The crack ran under his feet,  
as if tracing a choice he had not made.

Behind him, three fire-walkers walked the spiral.  
In silence.

Before him, banners were raised —  
five-colored, flame-edged, shaped like wings.

A child on the other side shouted,  
 “They say you used to rule!”

Priotheer did not answer.

The ground beneath his feet did.

It hummed —  
not in resonance,  
but in reminder.

A low, aching tone  
that said:  
 \textbf{This, too, will become memory.}

And the dust curled in a wind  
that carried no voice at all.

\end{document}

