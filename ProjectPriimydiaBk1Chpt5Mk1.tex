\documentclass[10pt]{article}
\usepackage[margin=1in]{geometry}
\usepackage{setspace}
\usepackage{parskip}
\usepackage{lmodern}
\usepackage{titlesec}
\titleformat{\section}{\normalfont\Large\bfseries}{\thesection.}{1em}{}

\begin{document}

\begin{center}
    \Large\textbf{Book I: The One and the Five} \\
    \large Chapter 5: The Oath Before Flame \\
\end{center}

\vspace{1in}


\begin{center}
\begin{enumerate}
    \item \textbf{Anytus Declares the Mandate} \\
    In the newly built city of Elsewhere, Anytus establishes the Mandate of the Five — a structure of order, division, and daily recitation. It is not built on memory, but on speech. Priotheer is referenced only as absence.

    \vspace{1in}
    \item \textbf{The Silence Between Stones} \\
    Priotheer walks through the old city unnoticed. Chants have become games. A Fire-Walker forgets part of the First Oath. Memory no longer echoes — it hums faintly beneath the dust.

    \vspace{1in}
    \item \textbf{The Voice That Said Too Much} \\
    A priest attempts to invoke all five gods in a single rite. He is silenced — not by decree, but by breath collapsing. Anytus contains the event. A Fire-Walker marks the ground with six stones: five named, one left blank.

    \vspace{1in}
    \item \textbf{The One Who Did Not Bow} \\
    Anytus calls Priotheer to bow before the gods. Priotheer steps forward, says nothing, and the central flame extinguishes itself. No defiance, only absence. The Fire-Walkers return to their spiral in silence.

    \vspace{1in}
    \item \textbf{A Name Left Unspoken} \\
    Priotheer’s name fades — not by force, but forgetting. A young Fire-Walker remembers only his rhythm. The Mandate continues. But beneath it, the stone cracks. Silence begins to return — slow, sacred, and unnamed.
\end{enumerate}
\end{center}


\newpage

\section*{Segment 1: Anytus Declares the Mandate}

Stones were set in grids, not spirals.  
Each corner bore a sigil:  
Isgrine’s flame, Orson’s gate, Aerun’s eye, Paludin’s tear, Inanius’ empty ring.

At the center rose a platform of five gates, each facing one realm.  
Above them, a single name: \textbf{The Mandate of the Five}.

Anytus declared it not a temple, but an instruction.  

Not a sanctuary, but a sentence made stone.  
 “We were not meant to wander.  
 The gods have made clear what the Pattern only implied.”

He wore no crown.  

But his word moved faster than sound.  

He held no title.  

But each syllable became law.

Banners hung from every arch — five-colored, flame-fringed, symbol-set.  

Children recited daily rites.  

No one walked the spiral paths.  

They walked in lines.

The old chants were not banned.  
They were simply no longer needed.

 “We are not bound by memory,” Anytus said,  
 “but aligned by flame.”

No one spoke Priotheer’s name.  
But his silence was often referenced.

 “Some would have us kneel to the unknown.  
 Some still wait for resonance.  
 But waiting is how echoes fade.”

A new city rose around the Mandate.  
Its roads did not wind.  
Its stones did not sing.  
But they stood, evenly spaced, as if listening had been replaced by arrival.

At the far edge of the platform, a gate was left unopened.  
Its sigil was blank.

No one asked why.  
No one crossed it.

Not yet.


\newpage

\section*{Segment 2: The Silence Between Stones}

Priotheer walked the old paths.  
They did not resist him.  
But they no longer remembered him either.

Stones that once echoed his step returned only dust.  
The spiral lines were faint beneath the moss.  
And children playing nearby used them as game-markers —  
not knowing they once aligned breath with breath.

He passed a well once used for song-healing.  
Now a market had risen around it.  
At its center, a vendor sold carvings of the gods’ faces,  
each with a name etched below in flame-script.

He stopped.  
Watched.

A child approached with a chant —  
but it was clipped, flattened, ornamental.

 “Flame above… something, something… pattern, yes?”

She laughed and spun.

Her friend offered correction:

 “No — it’s ‘Flame above, speak below / Teach what only fire knows.’”

The girl shrugged.  
 “They keep changing them.”

No one looked at Priotheer.  
No one recognized him.

At the edge of the market, he found a Fire-Walker — one of the few who still wore no symbol.

The boy nodded in silence.

They walked together for a time.  
No chant between them.  
Only rhythm of foot and pause.

Eventually, the boy spoke.

 “I forgot the third line.  
 Of the First Oath.”

Priotheer did not answer.  
He only stopped.  
And whispered it.

The boy repeated it.  
This time, slower.

They stood at the circle’s rim.  
The Mandate rose in the distance — crisp, perfect, unyielding.

 “They’re louder now,” the boy said.

Priotheer nodded.

 “Then we must remember quieter.”

They did not move for a while.  
And beneath them, the stones did not echo —  
but they did not turn away.

\newpage

\section*{Segment 3: The Voice That Said Too Much}

It began at the center of the Mandate.

A priest stood before the gates,  
robes of all five colors braided along his arms.  
His voice was calm. Trained.  
He had led rites before.

But today, he declared something new:

 “Let all five speak.  
 Let their wisdom braid as breath.”

He raised both hands.

And then, in sequence too quick to echo,  
he invoked them all:

 “Isgrine, flame of judgment, rise—  
Orson, gate of law, open—  
Aerun, speak the winds—  
Paludin, bear our grief—  
Inanius, hold what must not be named—”

His breath caught.

Not because he forgot.  
But because \textbf{something answered.}

For a moment, the air rippled — as if five winds tried to pass through one door.  
The banners above the gates twisted inward.  
The priest gasped — and fell silent.

Not fainting.  
Not struck down.  
Just… silenced.

His mouth moved.  
No sound followed.

He tried again.  
And again.

Each word collapsed before leaving.

Anytus was already moving.  
He crossed the platform, calm as stone.  
He placed a hand on the priest’s shoulder.

 “The gods do not demand blending.  
 They demand clarity.”

He turned to the gathering crowd.

 “Let no one fear.  
 The Pattern holds.  
 But do not confuse the voices.”

The banners were lowered.  
The priest was taken away — still mute.  
His name was not spoken again.

No song followed.  
No question asked.

But later that night, a Fire-Walker walked past the Mandate.  
And without speaking, he placed five stones on the eastern side —  
one for each name.

And above them, he left a sixth.  
Unmarked.

\newpage

\section*{Segment 4: The One Who Did Not Bow}

They gathered at the center of the Mandate.

Not for sacrifice.  
Not for chant.  
For confirmation.

Anytus had summoned Priotheer — not by edict,  
but by story.

 “He walks still,” the people said.  
 “He listens still.  
 Let him speak — or bow.”

And so Priotheer came.  
Alone.

The platform had been cleared.  
Five flame-braziers lit.  
Each god’s sigil hovered above the gates.

Anytus stood between them, arms outstretched.

 “Before the gods, we call on the one who kept silence.  
 To kneel is not surrender.  
It is alignment.”

No reply.

Priotheer stepped forward.  
Only once.

The braziers flickered.

The wind stilled.

Anytus lowered his arms.

 “Will you not bow?”

Priotheer did not move.  
He did not speak.

He closed his eyes.

And the center brazier — the shared one —  
the flame meant to represent all five —  
went out.

Not extinguished.  
Withdrawn.

The crowd did not murmur.  
No shame was shouted.

Just air.  
Still air.

Anytus stood very still.

He looked at the four remaining flames.

He said only:

 “Then let it be memory.”

Priotheer turned and walked away.  
No one followed.

But that night, across the eastern rim,  
twelve Fire-Walkers walked the spiral path in silence.  
Not in defiance.  
In recognition.

And though no stars fell,  
the wind changed.

Just slightly.

\newpage

\section*{Segment 5: A Name Left Unspoken}

They did not speak of him after.

Not because they were forbidden.  
But because the Mandate offered other names —  
simpler ones.  
Louder ones.

Soon, even the children who once traced spirals in dust  
began drawing flames instead.

One girl, youngest of the Fire-Walkers,  
stopped during evening passage.

 “What do we call the one who walked before?”

No one answered.

Not because they didn’t know.  
But because saying it felt… too much.

She nodded.

 “Then I will say what he didn’t.”

She pressed her hand to the dust.  
Paused.  
And took one step forward.

The others followed.

That night, a chant passed among the spirals —  
but it had no words.  
Only breath.

Only the rhythm of a path once walked.

At the Mandate, Anytus declared the gates sealed for winter.  
Rites were recited.  
Flames rekindled.  
Everything sounded complete.

But beneath the platform,  
where the spiral once touched the foundation,  
a crack had begun to widen.

Not fast.  
Not loud.

Just enough  
for silence to slip through.


\end{document}

