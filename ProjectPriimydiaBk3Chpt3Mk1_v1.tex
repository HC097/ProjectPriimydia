\documentclass[9pt]{article}
\usepackage[margin=1in]{geometry}
\usepackage{setspace}
\usepackage{parskip}
\usepackage{lmodern}
\usepackage{titlesec}
\titleformat{\section}{\normalfont\Large\bfseries}{\thesection.}{1em}{}

\begin{document}

\begin{center}
    \Large\textbf{Book III: The Stone That Looped the World} \\
    \large Chapter 3: The King Who Chose to be Forgotten\\
\end{center}

\vspace{1in}


\begin{enumerate}
    \item \textbf{The Last Command} \\
    Priotheer does not return to the Council. No decree is issued. No farewell given. His silence is mistaken for fatigue — then tacit approval. The ministers begin to rule in his absence, then forget he was meant to return. Authority shifts not with rebellion, but with time.

    \vspace{1em}
    \item \textbf{The Voyage That Went Nowhere} \\
    A trading vessel sets off from the southern coast, chasing winds no one has measured. Twelve days pass. It returns — not knowingly. The sailors disembark believing they have found a new land. But it is the same port. The same dock. The same sky. None notice.

    \vspace{1em}
    \item \textbf{The Map That Defined the World} \\
    A royal cartographer finalizes a grand map of the known realm. It shows the island of Priimydia surrounded by steady waters. There are no borders — only sea. “This is all there is,” he says. And no one argues. The map is copied. Taught. Sanctified.

    \vspace{1em}
    \item \textbf{The Tale No One Finishes} \\
    In a tavern near the coast, an old storyteller begins a tale about a king who once moved stars with silence. But he forgets the next line. The patrons grow quiet, then resume drinking. One child asks, “Was he real?” No one answers. The fire burns low.

    \vspace{1em}
    \item \textbf{The Word That Couldn’t Be Translated} \\
    Beneath the Stone Tree, Priotheer lays one final palm to the soil. A leaf falls beside him. He whispers a single word — heard by no one, written by no hand, remembered by no tongue. Then he stills. Not dead. Not divine. Just… unseen.

    \vspace{1em}
    \item \textbf{The Peace Without Memory} \\
    Priimydia enters a golden quiet. The harvests hold. The skies calm. The people live well, and ask nothing. The Spiral is forgotten. The Flame is legend. The Wall is unnamed. And the man who sealed the world vanishes — not into myth, but into absence.

\end{enumerate}



\newpage

\subsection*{Segment 1: The Last Command}

There was no proclamation.

No horn at sunrise. No messenger from the stone gates. No parchment sealed in wax and reverence. The king simply did not arrive.

They waited at first, as was custom. The High Council, formed by Priotheer to help govern the Priimydian people in his increasing absence, assembled beneath the vaulted canopy of the Forum Hall — a circle of twelve chairs, eleven filled. The twelfth, carved of elderwood and inlaid with silver roots, remained empty. No one sat there. No one ever had.

When the sun reached its noon mark, a clerk adjusted the ledger. “Attendance marked,” he said. His voice did not tremble.

The Council continued.

They spoke of fisheries in the east, of new irrigation structures, of the year’s grain quotas and their modest surplus. They passed decrees, drafted writs, and affixed their seals. No law forbade them from ruling without Priotheer. No one had written such a law because no one had ever imagined the need.

In the evening, after the council adjourned, one of the ministers — a woman named Alisera, who still remembered when Priotheer had led the armies against the gods — lingered by the elderwood chair.

She did not sit. She only touched its armrest, once, and whispered, “He’ll come when it matters.”

The chair, of course, did not reply.

Priotheer had not been seen in weeks. Not formally. Not in court. There were reports of him near the Tree, walking barefoot through the root paths, his hands behind his back, whispering into soil that held no crop. A shepherd claimed he saw the king speaking to a stone, then nodding as if it had answered.

Such stories were common now. They carried no weight. Reverence became habit. Then myth. Then something quieter.

What mattered was that the roads remained open, the ports functional, the weather mild. There was peace. Not the kind you swore oaths to protect, but the kind that settled in when no one remembered why they were afraid.

In the official records of that season, there is no note of Priotheer’s absence. The ledgers do not lie — but they do not speak, either. They record what is, not what changed.

Only in a single margin, faint and nearly lost to mold, is there a penciled phrase beside the roster:

\begin{quote}
\emph{“The throne grows colder, but the sun remains.”}
\end{quote}

No signature. No mark.

Just a line written during the quiet that followed the last command that was never given.

\newpage

\subsection*{Segment 2: The Voyage That Went Nowhere}

They named the ship \emph{Seeker’s Wake}. No special reason. The name had simply come to the captain in a dream, and the harbor registry had no objections. It was a small vessel — four hands, two weeks of rations, a shallow hold meant for spices or fish, not discovery.

They left from the southern dock at dawn. Sails caught wind that hadn’t been forecast. The sky was cloudless. The sea opened before them like a scroll no one had yet written on.

They did not think they were escaping anything. No decree compelled them to leave, no vision drove them. They simply wanted to see how far the water went.

Twelve days passed.

The sea remained calm. The wind constant. They charted carefully — knot by knot, marking every turn, every drift, every sight of gull or crest. On the eleventh night, they saw a light in the distance. By morning, they had made landfall.

It was a lush cove with a familiar shore. Palm-fringed. Pebbled paths. A stone arch with weathered carvings that one sailor swore she had seen in her youth.

They called it ``New South.'' Unloaded. Drank. Laughed.

Only when they climbed the bluff above the beach did one of them see the city.

There it was: Priimydia.

Their own city.

Unmistakable.

They stood on a hill they had not climbed, looking down at homes they had not left, and watched merchants they knew carrying baskets they remembered packing.

At first, they believed it was coincidence — a trick of resemblance, an uncharted colony. But when the captain walked to the docks and found her name still on the manifest, she said nothing. She simply sat down.

The others followed.

No report was filed. No expedition was blamed. When asked how the voyage went, they said only:

\begin{quote}
\emph{``The sea was smooth. The winds were fair.''}
\end{quote}

No one asked where they had gone.

Because everyone knew where they had returned.

\newpage

\subsection*{Segment 3: The Map That Defined the World}

They called him Aruthan the Deliberate.

Not because he was slow, but because he was exact. Every mark he made — whether on vellum or stone — followed long calculation. He walked every road himself. Measured each horizon by hand. Where most relied on copies and legends, Aruthan relied only on sight.

He had been commissioned to draw the definitive map of the known world.

It took him eight years.

He journeyed the southern cliffs, the northern forests, the fog coasts, the salt marshes east of the Stone Tree. He boarded trade vessels, followed nomads, wintered with fishermen. He returned with scrolls filled edge to edge with measurements and drawings. No embellishment. Just truth.

When he unveiled the final work, it stretched from floor to ceiling in the Great Hall of Records.

The continent of Priimydia.\\
Its mountains. Its rivers. Its inner seas.\\
And then: water.

Nothing else.

No other landmasses.\\
No border not circled by sea.

The officials praised its clarity. The scholars approved its precision. Even the merchants agreed it would simplify trade.

Only one priest objected.

``Where is the realm beyond the foam?'' she asked.

Aruthan blinked. ``There is none.''

The council chamber was silent for a time.

Finally, the priest murmured, ``Then we are alone.''

But no one responded. The meeting moved on.

The map was copied by hand and distributed to every city and port. The original was mounted in the Hall of Records behind crystal pane. In time, children were taught it as gospel. ``This is the world,'' their teachers said, and pointed.

The phrase took root.

\begin{quote}
\emph{``This is the world.''}
\end{quote}

Scribes repeated it. Sailors repeated it. Even the priests — slowly, uncertainly — began to echo it.

No one lied. No one hid anything.

There was nothing to hide.

There was only the world.

And it was shaped like the island that had never known it was sealed.


\newpage

\subsection*{Segment 4: The Tale No One Finishes}

It was late in the season, the kind of evening when firelight seemed thicker than usual, and every chair in the tavern near the coast leaned a little closer to the hearth.

Old Meras had a voice like dust and cider. He was known not for the accuracy of his tales, but the conviction with which he told them. Children adored him. Adults tolerated him. He drank free when the mood was generous and paid in stories when it wasn’t.

That night, he began again.

``There was once a king,'' he said, stirring the embers with a poker that wasn’t his. ``A quiet king. A king who knew the names of the stones beneath his feet.''

A boy near the fire leaned forward. ``What was his name?''

Meras smiled. ``He had many. But one of them was the one the world forgot.''

He let the silence stretch, the way storytellers do when they want the ale to keep flowing.

``They say he could move the stars if he walked in the right pattern. That he once held back the sea with a whisper. That when the gods made war on the people, he stood between them — and spoke, just once.''

``What did he say?'' asked a girl near the wall.

Meras opened his mouth.

Then closed it.

A moment passed.

The fire cracked. Someone refilled his cup. He looked at it as if seeing it for the first time.

``I... don’t remember,'' he said.

A few chuckled. One patron clapped him on the back. ``You’re getting old, Meras.''

``I suppose,'' he muttered. ``I suppose I am.''

The children asked for another story — the one about the glass bear or the endless ladder. He gave it, gladly, voice rising again as if the lapse had never occurred.

But later, as the tavern emptied and the night deepened, Meras sat alone by the dying fire. His cup untouched. His eyes fixed not on the hearth, but on something further off — a shape he could not name.

And when the embers gave their final pop, he whispered, to no one at all:

\begin{quote}
\emph{``I think I almost remembered him.''}
\end{quote}


\newpage

\subsection*{Segment 5: The Word That Couldn’t Be Translated}

The Stone Tree was shedding.

Its leaves did not fall in clusters, nor with the riot of autumn. They dropped one at a time — slow, deliberate, as if remembering something with each descent. The ground beneath it was a quiet mosaic of yellow and grey.

Priotheer stood at its base.

He had not spoken in days. Perhaps weeks. There was no one left to listen, and even the soil seemed saturated. His breath came slower now. Not labored — just old. Not weak — just ending.

He knelt, one hand pressed against the roots, the other flat on the soil.

He was not casting a spell.

He was anchoring something.

A memory, perhaps. A shape. A syllable the world no longer recognized.

No magic glowed. No wind stirred.

He whispered.

It was a single word. Soft, fragile. Spoken not in the language of Orfyd, nor in the ancient tongue of the One, nor even in the spiraled cadence of the Flame chants.

It was the language that had come before all of them.

A word with no letters. No translation. Only shape.

The roots shivered, then stilled.

The Tree did not respond.

But the world did.

Not with thunder. Not with revelation.

Only with stillness.

One leaf drifted down — not in haste, not in ritual, just as leaves do — and settled beside his open hand.

He did not reach for it.

He only looked at it, and closed his eyes.

That was all.

No death.

No coronation.

No myth.

Just a king kneeling beside the roots of what he had sealed, having spoken the final word — and knowing that no one would remember what it meant.


\newpage

\subsection*{Segment 6: The Peace Without Memory}

They called it peace.

The roads stayed clear through the winter. The harvests came early. Fishermen returned with nets too full to explain. Even the mountain snows melted gently, feeding the rivers without flood.

Children were born. Games were invented. Songs were written — none about fire, none about gods.

Temples remained, but fewer came. The candles burned shorter. The prayers grew simpler. Some began to forget why certain holidays were celebrated at all. They still celebrated them. It felt right.

One clerk noted that the stars had shifted slightly that season. He marked it in his logbook and moved on.

No new lands were charted. No wars declared. A generation passed, then another.

The old stories remained, but only in corners — told by tired men to children more curious than reverent. When asked what the world had once feared, the elders paused, then smiled.

``I suppose we feared the end.''

``And it didn’t come?'' the children asked.

``No,'' came the answer. ``Not exactly.''

The throne in the Forum Hall was eventually removed. Not dismantled — simply replaced with a table. Its absence was never discussed. The elderwood was repurposed into doorframes.

Hardly anyone remembered Priotheer.\\
No one cursed him.\\
No one blessed him.\\
He was not betrayed — only overwritten.

The Tree still stood.

It shimmered, sometimes, when no one was looking. Its roots reached deeper than anyone had mapped. Its leaves still fell, one at a time.

And beneath it, something sat.

Not a man.

Not a king.

Only a shape, silent in the earth’s turning.

And the world turned, never asking what it had cost to be made whole.


\newpage

\end{document}

