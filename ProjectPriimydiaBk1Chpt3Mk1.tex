\documentclass[10pt]{article}
\usepackage[margin=1in]{geometry}
\usepackage{setspace}
\usepackage{parskip}
\usepackage{lmodern}
\usepackage{titlesec}
\titleformat{\section}{\normalfont\Large\bfseries}{\thesection.}{1em}{}

\begin{document}

\begin{center}
    \Large\textbf{Book I: The One and the Five} \\
    \large Chapter 3: The Gifts That Became Hunger \\
\end{center}

\vspace{1in}


\begin{center}
\begin{enumerate}
    \item \textbf{The Coming of the Five Gods} \\
    The five gods emerge from the realms — not summoned, but remembered into presence. Their gifts are received with awe, and Priotheer warns: “Do not build altars.” They nod — and begin to build.

    \vspace{1in}
    \item \textbf{Blessings Given and Feasts Shared} \\
      The divine gifts enrich the city: harvests, songs, iron, and silence. Gratitude turns to habit. Worship begins to replace memory. Priotheer is displaced. A statue stands where the Circle of Listening once gathered.
    \vspace{1in}
    \item \textbf{The Jealousy of the Gods Grows} \\
     The gods begin to watch — and withhold. Altars multiply. The city bends toward deities. Priotheer becomes mythically misattributed. He buries a bird that forgets how to fly.
    \vspace{1in}
    \item \textbf{When Gods Begin to Speak} \\
    The gods speak softly — in phrases meant to comfort and persuade. Their words become doctrine. The Meletheia fall silent. Priotheer hears a whisper from beyond the Pattern: “Why not lead again?”
    \vspace{1in}
    \item \textbf{The Pattern Begins to Fray} \\
    Memory falters. Chants crack. Stones fail. The gods reshape reality. Priotheer whispers, “This is not what we were given.” The chamber listens. The birds move their wings — but no longer in rhythm.

\end{enumerate}
\end{center}

\newpage

\section*{Segment 1: The Coming of the Five Gods}

They did not fall from the sky.  
They rose from the deep memory of the world.

When the Demiurge broke itself into echoes, the five realms caught fragments of its intention —  
not thought, not flesh, but \textbf{charge}.

And from those fragments, they began to form.

Not all at once.  
Not in light.  
But slowly, as alignment bent too long in one direction.

First came \textbf{Isgrine}, from the heat-veins of Isfyd — a god of judgment, whose breath smelled of metal, whose voice cracked stone.  

Then \textbf{Orson}, formed from the weight of Orfyd — broad-shouldered, unmoving, his skin carved with forgotten covenants.  

Then \textbf{Aerun}, born of winds and waiting — silent, faceless, watching.  

Then \textbf{Paludin}, risen from the grieving waters of Palus — her robes soaked in memory, her eyes leaking songs no one remembered teaching her.  

And last, \textbf{Inanius}, who came without shape — the god of absence, who whispered into the void of naming itself.

The people of Priimydia did not summon them.  
But when they arrived, they were not refused.

The gods came bearing gifts.

Isgrine gave flame that tempered steel.  
Orson gave stone that remembered its shape.  
Aerun gave breath that carried thought across distances.  
Paludin gave water that mourned for the drinker.  
Inanius gave nothing — only silence, and it made men weep.

The Priests of Memory recorded none of it.  
They could not.  
The gods did not speak in chant, and the chambers did not echo their names.

But the people accepted the gifts.

Not in greed — not yet.  
But in awe.

Fields flourished.  
Weapons strengthened.  
Songs carried.  
The city grew taller.  
Faster.

And something ancient beneath the Tree bent further.

Priotheer watched.

He did not forbid the gifts.  
He could not.  
But he warned the Circle:

“Do not build altars to what was never meant to walk beside you.”

They nodded.

And began building altars.


\newpage

\section*{Segment 2: Blessings Given and Feasts Shared}

The gifts were not stolen.  

They were given — freely, and with open hands.

The fire of Isgrine was shared with the blacksmiths.  

The memory-stone of Orson passed to the masons.  
Aerun’s whispering wind was captured in copper horns,  

Paludin’s water pooled in quiet wells beneath the city.  

And even Inanius left marks —  
not with gifts, but with gaps,  
in which the artists began to hear themselves.

At first, there was only wonder.

The markets filled with ironwork that hummed faintly when struck.  
Messages leapt between cities on spiraled wind-plates.  

Water drawn from Paludin’s wells tasted of grief, but also of clarity.  

Children dreamed of light with no source.

And so the people gave thanks.

They held festivals. 

Not chants, but feasts — music, firelight, painted cloth.

Altars rose. Not tall, but frequent.

One to Isgrine behind the forge.  

One to Aerun near the spires.  

One — a bowl of silence — left outside the Temple of Silence itself.

The Priests of Memory did not interfere.  
They were not priests of gods.

But they began to walk slower.

Some began to ask:  
Should not the gods be honored in the Meletheia?

Others:  
Could not their names be sung, just once, for thanks?

Priotheer heard.

He did not scold.  
But he left the city more often.  
He began walking the outer paths — not to forget, but to listen elsewhere.

One day he passed a field where the wheat shimmered gold.  

A farmer bowed and said, “Isgrine has blessed the harvest.”

Priotheer asked,  
“What song did you sing?”

The farmer looked confused.  
“We did not sing. We poured oil on the stone.”

Priotheer said nothing.  
But when he walked away, the wheat did not move in the wind.

He returned to the Circle of Listening that evening, and found it empty.

In its place, someone had left a statue — not large,  
but carved with a flame at its crown.

\newpage

\section*{Segment 3: The Singing Temples and the Withheld Answer}

The gods had not spoken before.

They had only given.

But now they watched.

They watched who lit fires at which altars.  

They watched which songs rose from which streets.  

They watched who knelt — and who did not.

Isgrine’s flame flickered higher at certain names.  

Paludin’s wells ran shallower when forgotten. 

The wind around Aerun’s temples grew sharp when chants lapsed.

They did not strike.  

They withheld.

A messenger fell silent mid-sentence.  
A smith’s iron refused to cool.  

One night, a singer found her voice scattered across three octaves — no chant would settle her.

The Priests of Memory were asked to intervene. 

They declined.  

 “The Meletheia do not name gods,” they said. 
 
 “We preserve what was sung, not what is desired.”

But desire had taken hold.

In the northern tier, a shrine to Orson doubled in size.  

In the western gardens, a fountain bore Paludin’s likeness, weeping stone into water.

People began to argue — not about gifts,  
but about \textbf{which god loved the city most.}

Some said Isgrine, for she gave them strength.  
Others, Aerun, for his wind carried hope.  
A few claimed Inanius was the truest — for silence is incorruptible.

The old spirals of the city became directional.  
Streets curved toward altars now.  
Homes built according to the Tree’s geometry began to warp — ever so slightly — toward divine centers.

Priotheer watched.

He was no longer the center.  
Not in anger.  
In fact, he felt strangely light.

He walked alone more often.  
He whispered the Meletheia into stones and streams,  
as if trying to remind the world of what had come before even the gods.

But the stones no longer answered.

And one evening, as he passed beneath a torch newly lit for Isgrine,  
he heard a child say:

 “That is the First Flame.”

The parent corrected him:  
 “No — that is Isgrine’s fire. The old king walks in memory.”

Priotheer did not stop walking.  
But a bird overhead lost its flight mid-wingbeat,  
and fell into the dust at his feet.

He buried it in silence.

\newpage

\section*{Segment 4: When gods Begin to Speak}

The gods had watched.  
The gods had waited.  
Now they began to speak.

Not in thunder. Not in song.  
But in phrases small enough to believe.

Isgrine spoke first.  
Her voice was found in the hiss of the forge.  
 “Strength is love. To protect is to lead.”

Orson followed.  
His voice came through the stones when touched by calloused hands. 

 “What is old is owed. Remember the shape of your father’s house.”

Aerun spoke only through wind at thresholds.  

 “You are more than breath. You are bridge.”

Paludin whispered from water drawn in grief.

 “Weep well, for sorrow gives shape to the soul.”

And Inanius —  
he spoke through nothing.  
But once, a child fell asleep near an empty shrine and woke speaking truths no one had taught him.

The words were not laws.  
They were comforts.  
But comforts bend into structure, if heard too long.

The people began to quote.  
To inscribe.  
Not the Meletheia — but the words of the gods.

Stones were carved.  
Walls painted.  
Feasts opened with recitation.

The Circle of Listening grew quieter.  
The Priests of Memory still walked —  
but fewer listened.  
And some began to forget what their colors even meant.

Priotheer stood beneath the Tree.  
Not at its center — but beneath it.  
He pressed his hand to the bark and whispered the First Song.

The Tree did not reply.  
But the wind did.

It carried a whisper — not from the One,  
not from the Pattern,  
but from somewhere *just adjacent* to memory.

 “Why not lead again?”

Priotheer turned away.  
But the whisper returned the next day.

And the next.

Not with command.  
But with invitation.

And for the first time in many years,  
he dreamed —  
not of what was,  
but of what might yet be done.

\newpage

\section*{Segment 5: The Pattern Begins to Fray}

The gods had spoken.  
And the city had listened.

But the Pattern had not.

The Meletheia — once clear as stars in rhythm — began to bend. 

A chant remembered for centuries cracked on its fourth line.  

A well, blessed with Paludin’s sorrow, began to hum with a joy no one placed there.  

Stones once used for alignment no longer resonated — they simply echoed.

Priests of Memory adjusted their step.  
They rechecked the Tree’s concentric paths.  
They spoke the old lines with sharper breath, hoping to smooth the edges.  
But the edges grew teeth.

In the Temple of Silence, the whispering void once held sacred  
began to produce faint syllables —  
not language, but a suggestion of intent.

Inanius was said to be pleased.

At the city’s southern tier, two boys recited the same Meletheic line —  
but the air around them responded differently.  
One’s voice deepened. The other fell silent mid-chant,  
his tongue stilled for a week.

No one could explain it.  
Except to say:  
 “The gods listen now. And perhaps the Pattern must adapt.”

Priotheer returned to the Chamber of Refrain.  
He entered alone.  
He sat in its center and said nothing.

The chamber whispered.

It offered three voices.  
One was his own.  
One was the voice of the wind.  
The third — he could not name.

He stood.  
He touched the centerstone.

And he said aloud:

 “This is not what we were given.”

The chamber was still.

But one ring on the floor, long unlit,  
began to glow — not with fire,  
but with memory scraped raw.

The Circle of Listening met the next morning.

Priotheer said only:

 “The gods have given much.  
 And we are forgetting why we were ever given anything at all.”

Then he left.

And the birds did not sing that morning.  
But neither were they silent.  
They moved their wings without rhythm.

\end{document}

